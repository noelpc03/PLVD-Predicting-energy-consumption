\documentclass[12pt,a4paper]{article}
\usepackage[utf8]{inputenc}
\usepackage[spanish]{babel}
\usepackage{geometry}
\usepackage{setspace}
\usepackage{hyperref}
\usepackage{titlesec}

\geometry{margin=2.5cm}
\setstretch{1.3}

\titleformat{\section}{\large\bfseries}{\thesection.}{1em}{}

\begin{document}

%----------------------------------------------------------
% PORTADA
%----------------------------------------------------------
\begin{titlepage}
    \centering
    \vspace*{3cm}
    {\LARGE\textbf{Predicción de Consumo Energético en Ciudades Inteligentes}}\\[1cm]
    {\Large Proyecto de Procesamiento de Grandes Volúmenes de Datos}\\[2cm]
    {\large Autores:}\\[0.3cm]
    {\large [Amalia Beatriz Valiente Hinojosa]}\\
    {\large [Noel Pérez Calvo]}\\[1cm]    
    {\large Fecha: \today}\\[4cm]
    \vfill
\end{titlepage}

%----------------------------------------------------------
% CONTENIDO
%----------------------------------------------------------

\section{¿Qué se pretende lograr?}

El objetivo central del proyecto es \textbf{predecir el consumo energético en distintas zonas de una ciudad inteligente}, utilizando técnicas de procesamiento de grandes volúmenes de datos mediante las plataformas \textit{Hadoop} y \textit{Apache Spark}.  
A través del análisis de datos históricos de consumo eléctrico, se busca anticipar la demanda en las próximas horas, detectar picos de consumo y evaluar el impacto de las condiciones climáticas sobre el uso energético urbano.

\section{Dataset seleccionado}

\textbf{Nombre:} \textit{Household Electric Power Consumption Dataset} \\[0.2cm]
\textbf{Fuente:} Kaggle / UCI Machine Learning Repository \\[0.2cm]
\textbf{Formato:} CSV (valores separados por punto y coma “;”) \\[0.2cm]
\textbf{URL:} \href{https://www.kaggle.com/datasets/uciml/electric-power-consumption-data-set}{https://www.kaggle.com/datasets/uciml/electric-power-consumption-data-set}

\section{Justificación del dataset}

\subsection*{Volumen}
El conjunto de datos contiene más de \textbf{dos millones de registros}, correspondientes a mediciones minuto a minuto del consumo eléctrico de un hogar durante casi \textbf{cuatro años} (2006–2010).  
Su tamaño aproximado de 120 MB en formato txt, y el incremento que supone su almacenamiento distribuido en \textit{HDFS}, permiten simular un entorno realista de \textbf{procesamiento de grandes volúmenes de datos}, adecuado para la aplicación de tecnologías como \textit{Hadoop} y \textit{Spark}.

\subsection*{Características}
El dataset incluye variables como:
\begin{itemize}
    \item Fecha y hora de cada registro.
    \item Potencia activa y reactiva global.
    \item Voltaje y corriente.
    \item Energía submedida en tres zonas del hogar (\textit{Sub\_metering\_1}, \textit{2} y \textit{3}).
\end{itemize}
Estas variables conforman una \textbf{serie temporal multivariable}, adecuada para tareas de predicción y análisis de patrones de consumo.  
Además, los datos presentan una variabilidad natural y cierto nivel de ruido, lo que refleja condiciones reales de consumo y permite evaluar técnicas de limpieza y modelado robustas.

\subsection*{Pertinencia}
El conjunto de datos resulta altamente pertinente para los objetivos del proyecto, ya que:
\begin{itemize}
    \item Contiene \textbf{registros reales de consumo energético}, directamente relacionados con la meta de predecir la demanda eléctrica.
    \item Su granularidad temporal (minuto a minuto) facilita el análisis de \textbf{patrones horarios, diarios y estacionales}.
    \item Puede combinarse con datos meteorológicos (temperatura, humedad, precipitaciones) para analizar el \textbf{impacto del clima} en la demanda, fortaleciendo el enfoque de ciudades inteligentes.
\end{itemize}

\section*{Conclusión}
El dataset seleccionado constituye una base sólida para el desarrollo del proyecto, ya que ofrece volumen, calidad y relevancia suficientes para implementar un sistema de predicción energética escalable mediante herramientas de procesamiento distribuido.  
De esta manera, se busca contribuir al diseño de estrategias de eficiencia y sostenibilidad energética en el contexto de las ciudades inteligentes.

\end{document}
